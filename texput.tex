\title{An introduction to sounddevice}

\maketitle

\href{https://python-sounddevice.readthedocs.io}{sounddevice} is a
Python \href{https://docs.python.org/3/tutorial/modules.html}{module}
that provides bindings for the
\href{http://www.portaudio.com/}{PortAudio
  library}~\cite{sounddevice}.

Let's see some examples of what sounddevice can do:

\subsection*{Wiring the ADC and the DAC using a loop}
\lstinputlisting[language=Python]{/home/vruiz/intercom/test/sounddevice/wire5.py}

This module implements the algorithm:

\begin{lstlisting}[language=Python]
while True:
  chunk = sound_card.read(chunk_size = 1024)  # (1)
  sound_card.write(chunk)  # (2)
\end{lstlisting}
  
  %\begin{pseudocode}[display]{}{}
  %  \BEGIN
  %    \mathtt{CHUNK\_SIZE} \GETS 1024\\
  %    \WHILE \TRUE
  %    \BEGIN
  %      \mathtt{chunk} \GETS \mathtt{stream}.\mathtt{read}(\mathtt{CHUNK\_SIZE}) \STMTNUM{1in}{st.1}\\
  %      \mathtt{stream}.\mathtt{write}(\mathtt{chunk}) \STMTNUM{1.03in}{st.2}
  %    \END
  %  \END
  %\end{pseudocode}

where \texttt{(1)} captures 1024
\href{https://python-sounddevice.readthedocs.io/en/0.3.12/api.html}{frames}
from the ADC, and \texttt{(2)} plays the chunk of frames. In
\texttt{sounddevice} a frame is a collection of one or more samples
(typically, a frame is a single sample if the number of channels is 1,
or two samples if the number of channels is 2).

If you want to run this module right now and you are not using a
\href{https://en.wikipedia.org/wiki/Headset_(audio)}{headset}, check
first that the output volumen of your
\href{https://en.wikipedia.org/wiki/Headset_(audio)}{speakers} is not
too high, otherwise you could involuntary
\href{https://www.youtube.com/watch?v=rI90lhYAffo}{``couple''} the
speaker and the
\href{https://en.wikipedia.org/wiki/Microphone}{mic(rophone)} of your
computer, producing a loud and annoying
\href{https://www.cirrusresearch.co.uk/blog/2012/03/tonal-noise-analysis-with-the-optimus-green-sound-level-meters/}{tonal
  sound}. In order to mitigate this effect, you can also control the
gain of your mic (if the gain is 0, no feedback between the speaker
and the mic will be possible). In Xubuntu, these controls are
available through clicking in the speaker icon (situated in the
top-right corner of your screen) of the Xfce window manager.

To run the module:

\begin{lstlisting}[language=Bash]
python wire5.py
\end{lstlisting}

Stop (killing) the module by clicking the CTRL- and c-keys (CTRL+c),
simultaneously.

The chunk size introduces some latency. If we want to measure it:

\begin{enumerate}
\item First, we need the tools:
  \href{http://sox.sourceforge.net/}{SoX},
  \href{https://www.audacityteam.org/}{Audacity}, and
  \href{https://raw.githubusercontent.com/Tecnologias-multimedia/intercom/master/test/sounddevice/plot_input.py}{plot\_input.py}:
  
  \begin{lstlisting}[language=Bash]
sudo apt install sox
sudo apt install audacity
sudo apt install curl
curl https://raw.githubusercontent.com/Tecnologias-multimedia/intercom/master/test/sounddevice/plot_input.py > plot_input.py
  \end{lstlisting}
  
\item Run:
  
  \begin{lstlisting}[language=Bash]
pip install matplotlib
python plot_input.py
  \end{lstlisting}

  and check that the gain of the mic does not produce
  \href{https://en.wikipedia.org/wiki/Clipping_(audio)}{clipping}
  during the sound recording.

\item \label{start_point} In a terminal, run:

  \begin{lstlisting}[language=Bash]
python wire3.py
  \end{lstlisting}

  while you control the output volume of the speakers to produce a
  decaying coupling noisy effect between both devices (the speaker(s)
  and the mic). If your desktop has not these
  \href{https://en.wikipedia.org/wiki/Transducer}{transducers}, we can
  use a
  \href{https://www.google.com/search?q=male+to+male+audio+jack+cable&client=firefox-b-d&sxsrf=ALeKk00GZUDGqiOfc0D8xkA_MIYgCuZmSA:1600270049146&source=lnms&tbm=isch&sa=X&ved=2ahUKEwjdvsu-_u3rAhXl0eAKHS90DUoQ_AUoAXoECA0QAw&biw=4288&bih=972}{male-to-male
    jack audio cable} and connect the line-output of your soundcard to
  the input of your sound card.

\item In a different terminal (keep \texttt{python wire3.py} running),
  run:

  \begin{lstlisting}[language=Bash]
sox -t alsa default test.wav
  \end{lstlisting}

  to save the ADC's output to the file \texttt{test.wav}.

\item While \texttt{sox} is recording, produce some short sound (for
  example, hit your laptop or your micro with one or your nails). Do
  this at least a couple of times more, to be sure that you record the
  sound and also the feedback of such sound. It's important the sound
  to be short (a few miliseconds) in order to visually recognize it
  and it's replicas.

\item Stop \texttt{sox} by pressing the CTRL+c (at the same
  time). This kills \texttt{sox}.

\item Kill \texttt{python wire3.pt} with CTRL+c.

\item Load the sound file into Audacity:
    
  \begin{lstlisting}[language=Bash]
audacity test.wav
  \end{lstlisting}

\item Localize the first one of your hitting-nail sounds in the
  \href{https://manual.audacityteam.org/man/audio_tracks.html}{audio
    track} of Audacity.

\item Select (using the mouse) the region that contains your sound and
  a replica.

\item Use the
  \href{https://manual.audacityteam.org/man/edit_toolbar.html#zoom_to_selection}{zoom
    to selection buttom} to zoom-in the selected area.

\item Measure the time between the ocurrence of the hit (of the nail)
  and the recording of its first replica produced by the
  speaker-to-the-mic feedback. This time is the real latency of your
  computer runing \texttt{wire3.py}.

\item Modify the constant $\mathtt{CHUNK\_SIZE}$ in the module and
  repeat this process, starting at the Step \ref{start_point}. Create
  an ASCII file (named \texttt{latency\_vs\_chunk\_size.txt}) with the
  content (use TAB-ulators to space the columns):
\begin{verbatim}
# CHUNK_SIZE    real
32              ...
64              ...
128             ...
256             ...
512             ...
1024            ...
2048            ...
4096            ...
8192            ...
\end{verbatim}
with the real (practical) latency.
\end{enumerate}

At this point, we know the real latency of \texttt{wire3.py} as a
function of $\mathtt{CHUNK\_SIZE}$. Plot the file
\texttt{latency\_vs\_chunk\_size.txt} with:

\begin{lstlisting}[language=Bash]
 sudo apt install gnuplot
 echo "set terminal pdf; set output 'latency_vs_chunk_size.pdf'; set xlabel 'CHUNK\_SIZE (frames)'; set ylabel 'Latency (seconds)'; plot 'latency_vs_chunk_size.txt' title '' with linespoints" | gnuplot
\end{lstlisting}

Let's compute now the buffering latency of a chunk (the
chunk-time). If $\mathtt{sampling\_rate}$ is the number of frames per
second during the recording process, it holds that:
  
%\begingroup
%\Large
\begin{equation}
    %\displaystyle
  \mathtt{minimal\_buffering\_latency} = \mathtt{CHUNK\_SIZE} / \mathtt{sampling\_rate}
\end{equation}
  %\endgroup

Add these calculations to \texttt{latency\_vs\_chunk\_size.txt} using
  a third column (remember to use TABs).
\begin{verbatim}
# CHUNK_SIZE    real    minimal
:               :       :
\end{verbatim}

Plot both latencies:

\begin{lstlisting}[language=Bash]
echo "set terminal pdf; set output 'latency_vs_chunk_size.pdf'; set xlabel 'CHUNK\_SIZE (frames)'; set ylabel 'Latency (seconds)'; set key left; plot 'latency_vs_chunk_size.txt' using 1:2 title 'Real' with linespoints, 'latency_vs_chunk_size.txt' using 1:3 title 'Minimal' with linespoints" | gnuplot
\end{lstlisting}
  
Which seems to be the minimal practical (real) latency (the latency
obtained ideallly when $\mathtt{CHUNK\_SIZE}=1$ ... however, notice
that depending on your computer, this chunk size can be too small,
overwhelming the CPU) in your computer?  Justify your
answers. \label{question}

\subsection*{Wiring the ADC and the DAC using an interruption handler}
\lstinputlisting[language=Python]{/home/vruiz/intercom/test/sounddevice/wire.py}

\section{Resources}

\bibliography{sound}
